\thispagestyle{acknowledgements}
\section*{Abstract}
\label{sec:abstract}
\addcontentsline{toc}{section}{Abstract}

Self-Sovereign Identity (SSI) is a new approach on data privacy that puts data in the hands of its owner. These concepts were previously only tied to humans, and for this reason the assessment of the requirements to use these concepts to manage the digital identity of Internet of Thing (IoT) devices is a topic that needs further investigation. This study aims to understand the requirements that could enable an SSI-driven system with IoT devices. The focus is set on Electric Vehicle Charging Networks, and how the current system could be migrated to an approach that eliminates the identified privacy, security and transparency issues. To achieve this, a proof-of-concept was developed to assess how the new system would compare to the previous one, and understand the constraints introduced by this new approach. Making use of Hyperledger Indy and Aries, two projects centered around providing means for SSI, a novel system was built that allows users to hold their data in their own virtual wallets, which they can later use to present any claims needed by the system. While investigating the drawbacks of this system, the immaturity of the ecosystem was identified as the most pressing matter to effectively implement the envisioned system, as well as eliminating possibly introduced privacy issues. Aided by contributions from domain experts, a positive assessment of the system was made, with the unfulfilled requirements being outlined and with a road-map being designed to mitigate these flaws. 

%Ultimately, the use of SSI in the analysed use case provided good indications for the field of SSI with IoT.
% Ultimately, only one use case of SSI with IoT was studied, but it provides good indications of the field of SSI with IoT for future work.